% !TEX encoding = MacOSRoman
%---------------------------------------------------------------------
\documentclass{standalone}

\usepackage[applemac]{inputenc} % Mac
\usepackage[portuges,brazilian]{babel}
\usepackage[T1]{fontenc}
\usepackage{verbatim}
\usepackage{icomma}

\usepackage{tikz}
\usepackage{pgfplots}
\pgfplotsset{compat=newest}

\begin{document}
\begin{tikzpicture}[>=latex]

% -------------------------------------------------------------------------
% 1) Define basic math functions for Normal and Skew-Normal
% -------------------------------------------------------------------------
% Standard Normal PDF: phi(x) = 1/sqrt(2*pi)*exp(-x^2/2)
\pgfmathdeclarefunction{stdNormalPDF}{1}{%
  \pgfmathparse{1/sqrt(2*pi)*exp(-(#1)^2/2)}%
}

% Standard Normal CDF (using error function): 
% Phi(x) = 0.5 * [1 + erf(x / sqrt(2))]
\pgfmathdeclarefunction{stdNormalCDF}{1}{%
  \pgfmathparse{0.5*(1 + erf(#1/sqrt(2)))}%
}

% Skew-Normal PDF with shape alpha:
%   f_SN(x; alpha) = 2 * phi(x) * Phi(alpha*x)
\pgfmathdeclarefunction{skewNormal}{2}{%  args: x, alpha
  % phi(x):
  \pgfmathparse{1/sqrt(2*pi)*exp(-(#1)^2/2)}%
  \let\phi\pgfmathresult
  % Phi(alpha*x):
  \pgfmathparse{0.5*(1 + erf((#2*#1)/sqrt(2)))}%
  \let\Phi\pgfmathresult
  % combine:
  \pgfmathparse{2 * \phi * \Phi}%
}


% -------------------------------------------------------------------------
% 2) Define functions for "kurtosis" panel (just for illustration)
%    We'll overlay a Laplace-like shape (positive kurtosis) and
%    a super-Gaussian shape (negative kurtosis), both centered at 0
% -------------------------------------------------------------------------
% (A) Laplace(0,1) unnormalized (for illustration):
%     f(x) ~ exp(-|x|)
\pgfmathdeclarefunction{laplace}{1}{%
  \pgfmathparse{exp(-abs(#1))}%
}

% (B) Super-Gaussian shape (negative kurtosis), e^{- x^4} unnormalized:
\pgfmathdeclarefunction{superGaussian}{1}{%
  \pgfmathparse{exp(-((#1)^4))}%
}

% (C) Standard Normal again for reference:
%     we already have stdNormalPDF(x).

% -------------------------------------------------------------------------
% 3) First Panel: Negative Skew vs. Normal
% -------------------------------------------------------------------------
\begin{scope}[xshift=0cm]
  % Axes (optional)
  \draw[->] (-3.2,0) -- (3.2,0) node[below]{\small $x$};
  \draw (0,0) node[below left] {\small Panel (a)};

  % Normal in black
  \draw[domain=-3:3,samples=100,smooth,very thick]
       plot(\x,{stdNormalPDF(\x)}) 
       node[right,black]{\tiny};

  % Negative Skew-Normal in gray
  \draw[domain=-3:3,samples=100,smooth,very thick,gray!60]
       plot(\x,{skewNormal(\x,-4)}) 
       node[right,black]{\tiny};
\end{scope}

% -------------------------------------------------------------------------
% 4) Second Panel: Positive Skew vs. Normal
% -------------------------------------------------------------------------
\begin{scope}[xshift=6cm]
  % Axes (optional)
  \draw[->] (-3.2,0) -- (3.2,0) node[below]{\small $x$};
  \draw (0,0) node[below left] {\small Panel (b)};

  % Normal in black
  \draw[domain=-3:3,samples=100,smooth,very thick]
       plot(\x,{stdNormalPDF(\x)})
       node[right,black]{\tiny};

  % Positive Skew-Normal in gray
  \draw[domain=-3:3,samples=100,smooth,very thick,gray!60]
       plot(\x,{skewNormal(\x,4)})
       node[right,black]{\tiny};
\end{scope}

% -------------------------------------------------------------------------
% 5) Third Panel: Kurtosis Variation vs. Normal
%    We'll overlay:
%      - Normal in black
%      - Laplace-like (gray!60) for heavier tails (positive kurtosis)
%      - Super-Gaussian (gray!30) for negative kurtosis
% -------------------------------------------------------------------------
\begin{scope}[xshift=12cm]
  % Axes (optional)
  \draw[->] (-3.2,0) -- (3.2,0) node[below]{\small $x$};
  \draw (0,0) node[below left] {\small Panel (c)};

  % Normal in black
  \draw[domain=-3:3,samples=150,smooth,very thick]
       plot(\x,{stdNormalPDF(\x)})
       node[right,black]{\tiny};

  % Positive kurtosis (Laplace-like) in darker gray
  \draw[domain=-3:3,samples=150,smooth,very thick,gray!60]
       plot(\x,{laplace(\x)})
       node[right,black]{\tiny};

  % Negative kurtosis (super-Gaussian) in lighter gray
  \draw[domain=-3:3,samples=150,smooth,very thick,gray!30]
       plot(\x,{superGaussian(\x)})
       node[right,black]{\tiny};
\end{scope}

\end{tikzpicture}
\end{document}
%---------------------------------------------------------------------
% EOF